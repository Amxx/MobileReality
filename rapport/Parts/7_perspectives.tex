\documentclass[10pt,a4paper,twoside, twocolumn]{report}
%% Lots of packages !
\usepackage{etex}

%% Francisation
\usepackage[francais]{babel}
\usepackage[T1]{fontenc}
\usepackage[utf8]{inputenc}
%\usepackage{textcomp}

%% Réglages généraux
\usepackage[left=1.5cm,right=1.5cm,top=2cm,bottom=2cm]{geometry}
\usepackage{fancyhdr}
\usepackage{setspace}
\usepackage{lscape}
%\usepackage{multicol}
\usepackage{makeidx}
\usepackage[clearempty]{titlesec}
\usepackage{cite}

%% Packages pour le texte
\usepackage{pifont}
\usepackage{eurosym}
\usepackage{soul}
\usepackage[normalem]{ulem}
\usepackage{fancybox}
\usepackage{boxedminipage}
\usepackage{enumerate}
\usepackage{verbatim}
\usepackage{moreverb}
\usepackage{listings}
\usepackage[table]{xcolor}

%% Packages pour les tableaux
\usepackage{array}
\usepackage{multirow}
\usepackage{tabularx}
\usepackage{longtable}

%% Packages pour les dessins
\usepackage{graphicx}
\usepackage{wrapfig}
%\usepackage{picins}
\usepackage{picinpar}
\usepackage{epic}
\usepackage{eepic}
\usepackage{tikz}
\usepackage{afterpage}
\usepackage{rotating}
\usepackage{float}
\usepackage{caption}

%% Packages pour les maths
\usepackage{amsmath}
\usepackage{amssymb}
\usepackage{dsfont}
\usepackage{mathrsfs}
\usepackage{bussproofs}
\usepackage[thmmarks,amsmath]{ntheorem}

%% Création de nouvelles commandes
%\usepackage{calc}
\usepackage{ifthen}
\usepackage{xspace}



\usepackage{url}
\usepackage{hyperref}
\usepackage{todonotes}
\usepackage{subcaption}
\usepackage[french,ruled,vlined,linesnumbered,algochapter,dotocloa]{algorithm2e}
\usepackage{MnSymbol}

\usepackage{chngcntr}

\usepackage{standalone}
\usepackage{import}



\frenchbsetup{StandardEnumerateEnv=true}

%% =======================================================================

\fancypagestyle{empty}{%
  \fancyhf{}
  \fancyhead[L]{}
  \fancyhead[C]{}
  \fancyhead[R]{}
  \fancyfoot[L]{}
  \fancyfoot[C]{}
  \fancyfoot[R]{}
}
\fancypagestyle{basicstyle}{
	\fancyhf{}	
	\fancyhead[L]{}
	\fancyhead[C]{Rendu réaliste et temps réel pour la réalité augmentée}
	\fancyhead[R]{}
	\fancyfoot[L]{hadrien.croubois@ens-lyon.fr}
	\fancyfoot[C]{--~\thepage~--}
	\fancyfoot[R]{}
}
\pagestyle{basicstyle}

%% =======================================================================

\titleformat{\section}[frame]
{\normalfont}
{\filright \footnotesize \enspace Partie \thesection\enspace}
{6pt}
{\bfseries\filcenter}
	
\titleformat{\subsection}[frame]
{\normalfont}
{\filright \footnotesize \enspace \thesubsection\enspace}
{6pt}
{\filcenter}

\titleformat{\subsubsection}
{\titlerule \vspace{.8ex} \normalfont\itshape}
{\thesubsubsection}
{.5em}
{}

\titleformat{\chapter}[display]
{\normalfont\bfseries\filcenter}
{}
{1ex}
{\titlerule[2pt] \vspace{2ex} \LARGE}
[\vspace{1ex} {\titlerule[2pt]}]

\parindent=10pt
\DeclareUnicodeCharacter{00A0}{~}

%% =======================================================================



\newcommand{\HRule}{\rule{\linewidth}{0.5mm}}
\newcommand{\Hs}{\operatorname{HS}}




\floatstyle{ruled}
\restylefloat{figure}
\restylefloat{table}
\newfloat{code}{!h}{locode}{}
\floatname{code}{\textsc{code}}

\addto\captionsfrench{%
  \renewcommand{\listfigurename}{Liste des figures}%
  \renewcommand{\listtablename}{Liste des tableaux}%
  \renewcommand{\listalgorithmcfname}{Liste des algorithmes}%
}
\newcommand{\listofcode}{\listof{code}{Liste des codes}}

\numberwithin{code}{chapter}
\numberwithin{equation}{subsection}
\counterwithout{footnote}{chapter}






\newcommand{\framedgraphics}[2]{%
  \setlength{\fboxsep}{0pt}%
  \setlength{\fboxrule}{1pt}%
  \fbox{\includegraphics[{#1}]{{#2}}}%
}

\newcommand*{\captionsource}[2]{%
  \caption[{#1}]{%
    #1%
    \\\hspace{\linewidth}%
    \textbf{\textsc{Source}} #2%
  }%
}

\newcommand{\footurl}[2][]{\footnote{\textbf{#1}\href{#2}{#2}}}
% \newcommand{\footurl}[2][]{\footnote{\textbf{#1}\url{#2}}}







\newif\iftwocolumn
\twocolumntrue
\usetikzlibrary{3d,arrows, calc, backgrounds, petri, positioning, shadows, shapes}


\tikzset{
	persp/.style={scale=3.0,x={(-0.8cm,-0.4cm)},y={(0.8cm,-0.4cm)}, z={(0cm,1cm)}},
	points/.style={fill=white,draw=black,thick}
	grid/.style={very thin,gray},
	axis/.style={->,ultra thick},
	cube/.style={thick, fill=black!15,opacity=0.5},
	cube hidden/.style={dashed},
	block/.style={
		rectangle, rounded corners,
		draw=black!80,
		fill=black!10, fill opacity=0.5,
		text=black!90, text opacity=1.0,
    text height=1.5ex,
    text depth=.25ex,
    text width=6em,
    text centered
	}
}

\tikzstyle{class}			=[rectangle, rounded corners, draw=black, fill=blue!40, drop shadow, text centered, anchor=north, text=white,    text width=3cm]
\tikzstyle{module}		=[rectangle, rounded corners, draw=black, fill=red!40, 	drop shadow, text centered, anchor=north, text=white,    text width=3cm]
\tikzstyle{component}	=[rectangle, rounded corners, draw=black, fill=green,   drop shadow, text centered, anchor=north, text=black!90, text width=3cm]
\tikzstyle{single}		=[text height=1.5ex, text depth=0.25ex]
\tikzstyle{double}		=[text height=4.0ex, text depth=2.75ex]
\tikzstyle{triple}		=[text height=6.5ex, text depth=5.25ex]
\tikzstyle{quadru}		=[text height=9.0ex, text depth=7.75ex]
\newcommand*{\rootPath}{../}
\standalonetrue

\begin{document}


\iftwocolumn \twocolumn \else \onecolumn \fi


\chapter{Perspectives d’évolution}

Au delà des travaux développés au cours de ce stage, certains aspects sont encore en travaux. Les problématiques discutées dans ce chapitre sont sujet à modifications et pourraient faire l’objet de nouveaux travaux dans la continuité des résultats déjà obtenus.


\section{Acquisition HDR}

La méthode de rendu développée utilise les informations de luminosité de la scène. Ces informations, issues de caméra standard, ne traduisent pas la réalité physique de la luminosité de la scène mais plutôt la perception que l’on peut avoir de cette scène. Les données obtenues par le capteur photosensible sont en effet traitées pour obtenir une image classique.

L’imagerie HDR\footurl[HDR:~]{http://en.wikipedia.org/wiki/High-dynamic-range\_imaging} (High dynamic range), par opposition aux images classiques ou LDR (Low dynamic range), vise à considérer une dynamique de luminosité plus large qui permet de représenter les contrastes présents dans la nature. 

Les calculs d’éclairement développés dans ce rapport étant fait du point de vue de l’intensité lumineuse reçue par unité de surface, il faudrait théoriquement les appliquer à une envmap HDR qui reflète les fortes différences de luminosité présentes dans la scène. Cela implique d’acquérir, en temps réel, une vue HDR de la scène. 

Les méthodes d’acquisition HDR nécessitent généralement plusieurs prises de vues avec différentes expositions qui sont par la suite fusionnées en un unique image HDR. D’autres méthodes proposent, à défaut d’un grand nombres de prise de vue, d’utiliser différentes heuristiques pour reconstruire une image HDR à partir d’une unique image LDR\cite{Rempel2006}.

L’utilisation d’une envmap HDR devrait améliorer les résultats de rendu, notamment pour des scènes présentant de forts contrastes. Le modèle d’ombre douce présenté plus tôt devrait par ailleurs, avec une telle envmap, permettre de reconstruire des ombres dures dans le cas d’une source principale comme le soleil.

La principale difficulté vient de la reconstruction de l’image HDR en temps réel, notamment avec des caméras ne permettant pas de régler ou même de fixer les paramètres d’exposition.

La version $3.0$ d’OpenCV\footurl[OpenCV:~]{http://opencv.org/} devrait voir l’apparition d’outils pour la reconstruction et la manipulation d’images HDR. Il semble intéressant, une fois cette mise à jour disponible, d’étudier les possibilités offertes.


\section{BRDF anisotropes}

Une des perspectives évoquées dans la section rendu est la possibilité de considérer un modèle de surface à micro-facettes.

Un tel modèle, étoffé d’informations surfacique décrivant la distribution de micro-facettes, permet un rendu réaliste de matières anisotropes comme le métal brossé.

Pour cela de lourds calcules statistiques sont nécessaires dans l’évaluation de la direction de réflexion étant donné l’angle de vue et les caractéristiques de la distribution de micro-facettes au point considéré. Ces calculs statistiques devront par ailleurs être résolues de manière analytique, quitte à les approcher, au risque de trop ralentir l’étape de rendu.

La principal difficulté dans l’application d’un tel modèle reste cependant la nécessité d’évaluer l’envmap de manière anisotrope. Là où le filtrage linéaire permet d’intégrer l’envmap de manière isotrope de manière satisfaisante, le filtrage anisotrope de l’envmap pose problème au niveau des bordures entre les différentes faces constituants l’envmap cubique.


\section{Modèles animés}

Afin améliorer l’expérience de l’utilisateur, il pourrait être intéressant d’intégrer des modèles animés. Cela permettrait par exemple d’insérer des éléments dynamiques à une scène historique ou bien de d’afficher de manière vraisemblable des vêtements qui s’ajusteraient au client.

Afin de pouvoir insérer de tels objets, il est nécessaire de pouvoir adapter les données pré-calculées. 

Les sphères utilisées dans le calcul des ombres portées sont facilement adaptables. Il suffit de les poser sur un modèle de fil de fer qui représente les articulation de l’objet.

L’animation des textures d’auto-occultation à par ailleurs déjà été étudiée dans \cite{Kontkanen2006}

La plus grande difficulté reste à intégrer la gestion des objets animés et à mettre en place les éventuels mécanismes d’asservissement de l’animation par rapport à l’environnement.

%=====================================================================
%=====================================================================
\ifstandalone
	\addcontentsline{toc}{chapter}{Bibliographie}
	\bibliographystyle{apalike}
	\bibliography{\rootPath Annexes/biblio}
\fi
%=====================================================================
%=====================================================================
\end{document}