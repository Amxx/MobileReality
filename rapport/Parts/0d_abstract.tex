\documentclass[10pt,a4paper,twoside, twocolumn]{report}
\input{../Annexes/PackagesReport.tex}
\frenchbsetup{StandardEnumerateEnv=true}

\numberwithin{equation}{subsection}

%% =======================================================================

\fancypagestyle{empty}{%
  \fancyhf{}
  \fancyhead[L]{}
  \fancyhead[C]{}
  \fancyhead[R]{}
  \fancyfoot[L]{}
  \fancyfoot[C]{}
  \fancyfoot[R]{}
}
\fancypagestyle{basicstyle}{
	\fancyhf{}	
	\fancyhead[L]{}
	\fancyhead[C]{Rendu réalise et temps réel pour la réalité augmentée}
	\fancyhead[R]{}
	\fancyfoot[L]{hadrien.croubois@ens-lyon.fr}
	\fancyfoot[C]{--~\thepage~--}
	\fancyfoot[R]{}
}
\pagestyle{basicstyle}

%% =======================================================================

\titleformat{\section}[frame]
{\normalfont}
{\filright \footnotesize \enspace Partie \thesection\enspace}
{6pt}
{\bfseries\filcenter}
	
\titleformat{\subsection}[frame]
{\normalfont}
{\filright \footnotesize \enspace \thesubsection\enspace}
{6pt}
{\filcenter}

\titleformat{\subsubsection}
{\titlerule \vspace{.8ex} \normalfont\itshape}
{\thesubsubsection}
{.5em}
{}

\titleformat{\chapter}[display]
{\normalfont\bfseries\filcenter}
{}
{1ex}
{\titlerule[2pt] \vspace{2ex} \LARGE}
[\vspace{1ex} {\titlerule[2pt]}]

\parindent=10pt
\DeclareUnicodeCharacter{00A0}{~}

%% =======================================================================


\floatstyle{ruled}
\restylefloat{figure}
\restylefloat{table}
% \newfloat{code}{!h}{locode}{}
% \floatname{code}{\textsc{code}}


\addto\captionsfrench{%
  \renewcommand{\listfigurename}{Liste des figures}%
  \renewcommand{\listtablename}{Liste des tableaux}%
  \renewcommand{\listalgorithmcfname}{Liste des algorithmes}%
}
	









\newcommand{\HRule}{\rule{\linewidth}{0.5mm}}
\newcommand{\Hs}{\operatorname{HS}}




\newcommand{\framedgraphics}[2]{%
  \setlength{\fboxsep}{0pt}%
  \setlength{\fboxrule}{1pt}%
  \fbox{\includegraphics[{#1}]{{#2}}}%
}


\usetikzlibrary{3d,arrows, calc, backgrounds, petri, positioning, shapes.geometric}

\tikzset{
	persp/.style={scale=3.0,x={(-0.8cm,-0.4cm)},y={(0.8cm,-0.4cm)}, z={(0cm,1cm)}},
	points/.style={fill=white,draw=black,thick}
	grid/.style={very thin,gray},
	axis/.style={->,blue,ultra thick},
	cube/.style={thick, fill=black!15,opacity=0.5},
	cube hidden/.style={dashed},
	block/.style={
		rectangle, rounded corners,
		draw=black!80,
		fill=black!10, fill opacity=0.5,
		text=black!90, text opacity=1.0,
    text height=1.5ex,
    text depth=.25ex,
    text width=6em,
    text centered
	}
}

\newcommand*{\rootPath}{../}
\standalonetrue

\begin{document}

\vfill

\phantomsection
\addcontentsline{toc}{section}{Abstract}
\section*{Abstract}


  \vfill
  
	La réalité augmentée offre aujourd'hui de nouvelles possibilités d’interaction avec les utilisateurs. Le développement d'applications mobiles reste aujourd'hui cependant limité à des méthodes de rendu simples et ne permettant pas de résultats visuellement vraisemblables en temps réel. Nous développerons ici un pipeline de rendu permettant l’intégration d'objets de manière réaliste dans un environnement lumineux dynamique acquis en temps réel. L'utilisation des niveaux de détail des textures caractérisant l'environnement nous permet un rendu rapide des objets considerés ainsi que de leurs impact sur la scèene réel. Une telle solution logiciel à de nombreuses applications dans amélioration de l’expérience des utilisateurs de terminaux mobiles.

	\ifllncs
	Les consignes de l'université Claude Bernard Lyon I exigent de presenter ce document au format \texttt{llncs}. Ce format peut nuir à la lisibilité des figures presentes en annexe. Une autre mise en page du meme document, incluant une table des matière, est disponible dans la section \textsc{work} de mon site:

	\begin{center}\href{http://perso.ens-lyon.fr/hadrien.croubois}{http://perso.ens-lyon.fr/hadrien.croubois}\end{center}
	\fi

	\vspace{1cm}

	
	The development of augmented reality offers many improvements on how users interact with their mobile devices. However, today's mobile applications only offer limited performances when rendering objects into a real scene in real time as they need either complexe acquisition device or intensive computation. This paper develops a method for rendering virtual objects into a real scene, with dynamically acquired lightning, in real time, using mipmap levels for efficiently computing ligntning and shadows informations. Such a method opens many perspectives and help users to better interact with their mobile devices

	\ifllncs
	Because of university policy this document has been formated using \texttt{llncs} formating. This format impact the formating of large figures. Another version of the same document is available on my website:

	\begin{center}\href{http://perso.ens-lyon.fr/hadrien.croubois}{http://perso.ens-lyon.fr/hadrien.croubois}\end{center}
	\fi

  \vfill


\end{document}
