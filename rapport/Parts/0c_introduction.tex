\documentclass[10pt,a4paper,twoside, twocolumn]{report}
\input{../Annexes/PackagesReport.tex}
\frenchbsetup{StandardEnumerateEnv=true}

\numberwithin{equation}{subsection}

%% =======================================================================

\fancypagestyle{empty}{%
  \fancyhf{}
  \fancyhead[L]{}
  \fancyhead[C]{}
  \fancyhead[R]{}
  \fancyfoot[L]{}
  \fancyfoot[C]{}
  \fancyfoot[R]{}
}
\fancypagestyle{basicstyle}{
	\fancyhf{}	
	\fancyhead[L]{}
	\fancyhead[C]{Rendu réalise et temps réel pour la réalité augmentée}
	\fancyhead[R]{}
	\fancyfoot[L]{hadrien.croubois@ens-lyon.fr}
	\fancyfoot[C]{--~\thepage~--}
	\fancyfoot[R]{}
}
\pagestyle{basicstyle}

%% =======================================================================

\titleformat{\section}[frame]
{\normalfont}
{\filright \footnotesize \enspace Partie \thesection\enspace}
{6pt}
{\bfseries\filcenter}
	
\titleformat{\subsection}[frame]
{\normalfont}
{\filright \footnotesize \enspace \thesubsection\enspace}
{6pt}
{\filcenter}

\titleformat{\subsubsection}
{\titlerule \vspace{.8ex} \normalfont\itshape}
{\thesubsubsection}
{.5em}
{}

\titleformat{\chapter}[display]
{\normalfont\bfseries\filcenter}
{}
{1ex}
{\titlerule[2pt] \vspace{2ex} \LARGE}
[\vspace{1ex} {\titlerule[2pt]}]

\parindent=10pt
\DeclareUnicodeCharacter{00A0}{~}

%% =======================================================================


\floatstyle{ruled}
\restylefloat{figure}
\restylefloat{table}
% \newfloat{code}{!h}{locode}{}
% \floatname{code}{\textsc{code}}


\addto\captionsfrench{%
  \renewcommand{\listfigurename}{Liste des figures}%
  \renewcommand{\listtablename}{Liste des tableaux}%
  \renewcommand{\listalgorithmcfname}{Liste des algorithmes}%
}
	









\newcommand{\HRule}{\rule{\linewidth}{0.5mm}}
\newcommand{\Hs}{\operatorname{HS}}




\newcommand{\framedgraphics}[2]{%
  \setlength{\fboxsep}{0pt}%
  \setlength{\fboxrule}{1pt}%
  \fbox{\includegraphics[{#1}]{{#2}}}%
}


\usetikzlibrary{3d,arrows, calc, backgrounds, petri, positioning, shapes.geometric}

\tikzset{
	persp/.style={scale=3.0,x={(-0.8cm,-0.4cm)},y={(0.8cm,-0.4cm)}, z={(0cm,1cm)}},
	points/.style={fill=white,draw=black,thick}
	grid/.style={very thin,gray},
	axis/.style={->,blue,ultra thick},
	cube/.style={thick, fill=black!15,opacity=0.5},
	cube hidden/.style={dashed},
	block/.style={
		rectangle, rounded corners,
		draw=black!80,
		fill=black!10, fill opacity=0.5,
		text=black!90, text opacity=1.0,
    text height=1.5ex,
    text depth=.25ex,
    text width=6em,
    text centered
	}
}

\newcommand*{\rootPath}{../}
\standalonetrue

\begin{document}

\phantomsection
\addcontentsline{toc}{chapter}{Introduction}
\chapter*{Introduction}

Le travail détaillé dans ce document à été réalisé de février à juillet 2014 au sein de l'équipe R3AM\footnote{\url{http://liris.cnrs.fr/r3am/}} en vue de l’obtention du Master IGI de l'UCBL\footnote{\url{http://www.univ-lyon1.fr/}}. Il a été encadré par Jean-Philippe Farrugia et Jean-Claude Iehl

Ce travail à été financé par l'ENS de Lyon\footnote{\url{http://www.ens-lyon.eu/}}, au même titre que le reste du M2 IGI, dans le cadre de la troisième année correspondant au statut de normalien.

Ce stage est issue du constat selon lequel les applications orientées temps réel actuellement disponible sur mobile ne correspondent pas à ce que propose l'état de l'art en terme en terme de rendu réaliste. Une autre constatation est que les dispositif mobiles actuelles sont de plus en plus à même d’acquérir leur environnement (notamment au moyen de camera frontales).

L'optique de se stage est donc d'évaluer les possibilités d'acquisition et de rendu en temps réelle sur une plate-forme mobile et de proposer une solution logicielle complète mettant en œuvre les mécanismes considérés, du repérage à l'étape de rendu en passant par la reconstruction de l'environnement lumineux.

%=====================================================================
%=====================================================================
\ifstandalone
	\addcontentsline{toc}{chapter}{Bibliographie}
	\bibliographystyle{apalike}
	\bibliography{\rootPath Annexes/biblio}
\fi
%=====================================================================
%=====================================================================
\end{document}