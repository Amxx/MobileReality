\documentclass[10pt,a4paper,twoside, twocolumn]{report}
\input{../Annexes/PackagesReport.tex}
\frenchbsetup{StandardEnumerateEnv=true}

\numberwithin{equation}{subsection}

%% =======================================================================

\fancypagestyle{empty}{%
  \fancyhf{}
  \fancyhead[L]{}
  \fancyhead[C]{}
  \fancyhead[R]{}
  \fancyfoot[L]{}
  \fancyfoot[C]{}
  \fancyfoot[R]{}
}
\fancypagestyle{basicstyle}{
	\fancyhf{}	
	\fancyhead[L]{}
	\fancyhead[C]{Rendu réalise et temps réel pour la réalité augmentée}
	\fancyhead[R]{}
	\fancyfoot[L]{hadrien.croubois@ens-lyon.fr}
	\fancyfoot[C]{--~\thepage~--}
	\fancyfoot[R]{}
}
\pagestyle{basicstyle}

%% =======================================================================

\titleformat{\section}[frame]
{\normalfont}
{\filright \footnotesize \enspace Partie \thesection\enspace}
{6pt}
{\bfseries\filcenter}
	
\titleformat{\subsection}[frame]
{\normalfont}
{\filright \footnotesize \enspace \thesubsection\enspace}
{6pt}
{\filcenter}

\titleformat{\subsubsection}
{\titlerule \vspace{.8ex} \normalfont\itshape}
{\thesubsubsection}
{.5em}
{}

\titleformat{\chapter}[display]
{\normalfont\bfseries\filcenter}
{}
{1ex}
{\titlerule[2pt] \vspace{2ex} \LARGE}
[\vspace{1ex} {\titlerule[2pt]}]

\parindent=10pt
\DeclareUnicodeCharacter{00A0}{~}

%% =======================================================================


\floatstyle{ruled}
\restylefloat{figure}
\restylefloat{table}
% \newfloat{code}{!h}{locode}{}
% \floatname{code}{\textsc{code}}


\addto\captionsfrench{%
  \renewcommand{\listfigurename}{Liste des figures}%
  \renewcommand{\listtablename}{Liste des tableaux}%
  \renewcommand{\listalgorithmcfname}{Liste des algorithmes}%
}
	









\newcommand{\HRule}{\rule{\linewidth}{0.5mm}}
\newcommand{\Hs}{\operatorname{HS}}




\newcommand{\framedgraphics}[2]{%
  \setlength{\fboxsep}{0pt}%
  \setlength{\fboxrule}{1pt}%
  \fbox{\includegraphics[{#1}]{{#2}}}%
}


\usetikzlibrary{3d,arrows, calc, backgrounds, petri, positioning, shapes.geometric}

\tikzset{
	persp/.style={scale=3.0,x={(-0.8cm,-0.4cm)},y={(0.8cm,-0.4cm)}, z={(0cm,1cm)}},
	points/.style={fill=white,draw=black,thick}
	grid/.style={very thin,gray},
	axis/.style={->,blue,ultra thick},
	cube/.style={thick, fill=black!15,opacity=0.5},
	cube hidden/.style={dashed},
	block/.style={
		rectangle, rounded corners,
		draw=black!80,
		fill=black!10, fill opacity=0.5,
		text=black!90, text opacity=1.0,
    text height=1.5ex,
    text depth=.25ex,
    text width=6em,
    text centered
	}
}

\newcommand*{\rootPath}{../}
\standalonetrue

\begin{document}

\phantomsection
\chapter{Introduction}

Le travail détaillé dans ce document à été réalisé de février à juillet 2014 au sein de l'équipe R3AM\footurl[R3AM:~]{http://liris.cnrs.fr/r3am/} en vue de l’obtention du Master IGI de l'UCBL\footurl[UCBL:~]{http://www.univ-lyon1.fr/}. Il a été encadré par Jean-Philippe Farrugia et Jean-Claude Iehl.

Ce travail a été financé par l'ENS de Lyon\footurl[ENS de Lyon:~]{http://www.ens-lyon.eu/}, au même titre que le reste du M2 IGI, dans le cadre de la troisième année correspondant au statut de normalien.

Ce stage est issu du constat selon lequel les applications orientées temps réel actuellement disponible sur mobile ne correspondent pas à ce que propose l'état de l'art en terme de rendu réaliste. Une autre constatation est que les dispositifs mobiles actuels sont de plus en plus à même d’acquérir leur environnement (notamment au moyen de caméras frontales). 

La réalité augmentée consiste à intégrer, d'une manière ou d'une autre, des données numériques dans une scène réelle. Les problématiques soulevées sont très nombreuses afin de permettre une intégration vraisemblable de l'objet virtuelle.

Là où l'industrie cinématographique réalise aujourd'hui des effets visuelles tout à fait vraisemblable au prix de nombreuses heures de travail de la part d'infographistes et de machines spécialement dédiées au rendu, nous nous intéresserons à l'aspect “temps réel” de cette discipline. Le temps réel pose des problèmes aussi bien pour l'acquisition que pour le rendu. Il vas, en effet, falloir à la fois reconnaître la scène, notamment en terme d'environnement lumineux et de géométrie, pour pouvoir dans un second temps intégrer un objet virtuel à cette dernière. Un tel ajout implique de modéliser les interactions lumineuses correspondant à la fois à l'éclairage de l'objet par son environnement mais aussi à l'impact du dit objet, notamment en terme d'ombre portées, sur la scène.

Les deux problématiques sont aujourd'hui bien maîtrisées mais exploitent des méthodes qui demande des temps de traitement excluant le temps réel, à fortiori sur des terminaux mobiles disposant d'une puissance de calcul limité.

L'optique de se stage est donc d'évaluer les possibilités d'acquisition et de rendu en temps réelle sur une plate-forme mobile et de proposer une solution complète de rendu, en temps réel, à partir de données de carte d'environnement reconstruites dynamiquement. L'optique étant une intégration visuellement pertinente on s’intéressera plus à la vraisemblance des données qu'a leur exactitude physique, ce qui nous permettra de nous affranchir des méthodes de rendu statistiques très lourdes à exécuter.

Le code développée devra par ailleurs être conçu dans l'optique d'être facilement portable sur différentes plate-formes mobiles.


%=====================================================================
%=====================================================================
\ifstandalone
	\addcontentsline{toc}{chapter}{Bibliographie}
	\bibliographystyle{apalike}
	\bibliography{\rootPath Annexes/biblio}
\fi
%=====================================================================
%=====================================================================
\end{document}